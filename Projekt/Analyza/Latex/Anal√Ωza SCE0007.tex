\documentclass[11pt]{article}
\usepackage[T1]{fontenc}
\usepackage[utf8]{inputenc}
\usepackage[czech]{babel}
\usepackage{a4wide}
\usepackage{graphicx}
\usepackage{float}
\usepackage{lmodern}
\usepackage{listings}
\usepackage{color}
\usepackage{multicol}

\graphicspath{ {C:/Users/Michal/Dropbox/School/4. semestr/DAIS/Projekt/Images/} }

\definecolor{dkgreen}{rgb}{0,0.6,0}
\definecolor{gray}{rgb}{0.5,0.5,0.5}
\definecolor{mauve}{rgb}{0.58,0,0.82}

\lstset{
  language=SQL,
  aboveskip=3mm,
  belowskip=3mm,
  showstringspaces=false,
  columns=flexible,
  basicstyle={\small\ttfamily},
  numbers=none,
  numberstyle=\tiny\color{gray},
  keywordstyle=\color{blue},
  commentstyle=\color{dkgreen},
  stringstyle=\color{mauve},
  breaklines=true,
  breakatwhitespace=true,
  tabsize=4,
  linewidth=12cm
}

\begin{document}

%{\fontfamily{lmss}\selectfont

\begin{titlepage}
    \begin{center}

        \begin{figure}[H]
            \includegraphics[width=\textwidth]{logo.png}
            \centering
        \end{figure}
 
        {\LARGE Semestrální projekt: Databázové a informační systémy}\\
        \vspace{0.25cm}
        {\Large \textbf{Informační systém pro zjištění vlakových spojení}}
 
        \vspace{1.5cm}

        Vysoká škola báňská – Technická univerzita Ostrava\\
        Fakulta elektrotechniky a informatiky\\
        
        \vfill
        \begin{multicols}{2}
            \begin{flushleft}
                Databázové a informační systémy, 2019/2020\\
                Cvičící: doc. Ing. Radim Bača, Ph.D.\\
                Cvičení: ÚT 8:00 – 10:30\\
            \end{flushleft}
            \begin{flushright}
                \textbf{Michal Ščepka}\\
                \textbf{SCE0007}\\
                Ostrava, 2020
            \end{flushright}
        \end{multicols}
    \end{center}
\end{titlepage}


\tableofcontents
\newpage

\section{Specifikace zadání}

\subsection{Proč?}

Potřebujeme informační systém pro zjištění vlakových spojení. Systém má zjednodušit vyhledávání nejrychlejšího či nejlevnějšího vlakového spojení a usnadnit objednávání jízdenek.

\subsection{Kdo?}
Hlavní rolí bude \textbf{správce drah (admin)} což je osoba, která má na starost správu uživatelů. Může vytvořit uživatele \textbf{vlaková společnost}, který může přidávat nebo upravovat záznamy patřící jeho společnosti. Dále bude v systému uživatel \textbf{zákazník}, který si může prohlížet nabídky společností a objednat si jízdenku. Souhrnně budeme všechny role v systému nazývat \textbf{uživatel}.

\subsection{Vstupy}
Celý systém se bude týkat zejména tras a vlakových spojů na nich. Spoj se skládá z příjezdů do stanic. U spoje nás bude zajímat název spoje, vlaková společnost, cena za jeden ujetý kilometr, kapacita míst a pravidelnost. Příjezd bude obsahovat stanici, spoj, čas příjezdu, pořadí příjezdu a vzdálenost od startovní stanice. U stanice nás bude zajímat název stanice a město, ve kterém se nachází. Jízda bude obsahovat spoj a datum, kdy se bude konat. Jízdenka obsahuje uživatele, kterému patří, seznam jízd a vypočtenou cenu. V seznamu jízd v jízdence nás bude zajímat nástupní a výstupní stanice ke každé jízdě. Nový spoj, příjezd a jízdu může vytvořit pouze vlaková společnost.

U uživatele nás bude zajímat login, jméno a příjmení, emailová adresa, typ uživatele a čas poslední návštěvy v systému. Uživatel může mít objednaných mnoho jízdenek.

\subsection{Výstupy}
Hlavní výstup, který bude dostupný všem uživatelům je zobrazení spoje: název spoje, nástupní a výstupní stanice a město, stanice a města přes které spoj vede, čas odjezdu a příjezdu, cena jízdenky, vlaková společnost a počet volných míst ve vlaku. Dále si uživatel bude moci zobrazit svůj login, jméno, příjmení, emailovou adresu, čas poslední návštěvy v systému, seznam jeho objednaných jízdenek a historii jízdenek kterými cestoval.

Správce drah bude mít dostupný přehled spojů a jízd všech vlakových společností narozdíl od uživatelů typu vlaková společnost, kteří budou mít dostupné jen své jízdy a spoje.

\subsection{Funkce}
Správce drah bude mít možnost mazat uživatele v systému. Každý uživatel může aktualizovat svoje údaje a objednat/zrušit jízdenku. Vlaková společnost může upravovat nebo přidávat své spoje, jízdy a příjezdy. Systém neumožní přístup k operacím, které nejsou pro danou roli uživatele povoleny. Uživatel si nebude moct objednat jízdenku do plného vlaku. Systém bude sledovat změnu atributu cena za ujetý kilometr. Systém automaticky přepočítá cenu jízdenky po přidání jízdy do jejího seznamu jízd. Systém vypočte cenu jízdy na základě vzdálenosti ze startovní do cílové stanice a ceně za ujetý kilometr.

\subsection{Příklad výstupu}
\begin{figure}[H]
    \includegraphics[width=\textwidth]{vystup.png}
    \centering
    \caption{Příklad výstupu}
    \label{vystup}
\end{figure}

\newpage

\section{Datový model}

\subsection{Konceptuální model}
\begin{figure}[H]
    \includegraphics[width=\textwidth]{modry.png}
    \centering
    \caption{Konceptuální model}
    \label{konceptualni}
\end{figure}

\subsection{Relační model}
\begin{figure}[H]
    \includegraphics[width=\textwidth]{zluty.png}
    \centering
    \caption{Relační model}
    \label{relacni}
\end{figure}

\newpage

\subsection{Lineární zápis}
Legenda: \textbf{Tabulka}, \underline{primární klíč}, \textit{cizí klíč}, atribut\\
\noindent
\textbf{Mesto} (\underline{mesto\_id}, nazev, kraj)\\
\textbf{Stanice} (\underline{stanice\_id}, nazev, \textit{mesto\_id})\\
\textbf{Prijezd} (\textit{stanice\_id}, \textit{spoj\_id}, cas, poradi, vzdalenost)\\
\textbf{Spoj} (\underline{spoj\_id}, nazev, cena\_za\_km, kapacita\_mist, pravidelny, \textit{spolecnost\_id})\\
\textbf{Spolecnost} (\underline{spolecnost\_id}, nazev, web, email)\\
\textbf{Jizda} (\underline{jizda\_id}, datum\_start, datum\_cil, \textit{spoj\_id})\\
\textbf{Jizdenka\_Jizda} (\textit{jizdenka\_id}, \textit{jizda\_id}, \textit{stanice\_id\_start}, \textit{stanice\_id\_cil}, poradi)\\
\textbf{Jizdenka} (\underline{jizdenka\_id}, \textit{uživatel\_id}, cena)\\
\textbf{Uzivatel} (\underline{uzivatel\_id}, login, jmeno, prijmeni, email, typ, posledni\_navsteva)\\
\textbf{Historie\_ceny} (\underline{history\_id}, cena, datum, \textit{spoj\_id})\\

\subsection{Datový slovník}

Tabulka \textbf{Mesto}

\begin{table}[H]
    \begin{tabular}{|l|c|c|c|c|c|c|c|} \hline
                & Dat. Typ  & Délka & Klíč      & Null  & Index & IO    & Popis \\ \hline
    mesto\_id   & Integer   &       & Primární  & ne    & A     &       & Primární klíč \\ \hline
    nazev       & Varchar   & 30    &           & ne    &       &       & Název města \\ \hline
    kraj        & Varchar   & 30    &           & ne    &       &       & Kraj \\ \hline
    \end{tabular}
\end{table}

\noindent
Tabulka \textbf{Stanice}

\begin{table}[H]
    \begin{tabular}{|l|c|c|c|c|c|c|c|} \hline
                & Dat. Typ  & Délka & Klíč      & Null  & Index & IO    & Popis \\ \hline
    stanice\_id	& Integer	&       & Primární	& ne	& A		&       & Primární klíč \\ \hline
    nazev	    & Varchar	& 30	&           & ne	&		&       & Název stanice \\ \hline
    mesto\_id	& Integer	&       & Cizí	    & ne	&		&       & Město \\ \hline
    \end{tabular}
\end{table}

\noindent
Tabulka \textbf{Prijezd}

\begin{table}[H]
    \begin{tabular}{|l|c|c|c|c|c|c|c|} \hline
                & Dat. Typ  & Délka & Klíč              & Null  & Index & IO    & Popis \\ \hline
    stanice\_id	& Integer	&	    & Primární, Cizí	& ne	& A     &		& Stanice \\ \hline
    spoj\_id	& Integer	&	    & Primární, Cizí	& ne	& A	    &	    & Spoj \\ \hline
    cas	        & Time		&	    &                   & ne	&		&       & Čas příjezdu \\ \hline
    poradi	    & Integer	&		&                   & ne	&		&       & Pořadí příjezdu \\ \hline
    vzdalenost	& Integer	&		&                   & ne	&	    & 2     & Vzdálenost od\\ &&&&&&& startovní stanice \\ \hline
    \end{tabular}
\end{table}

\newpage

\noindent
Tabulka \textbf{Spoj}

\begin{table}[H]
    \begin{tabular}{|l|c|c|c|c|c|c|c|} \hline
                    & Dat. Typ  & Délka & Klíč      & Null  & Index & IO    & Popis \\ \hline
    spoj\_id	    & Integer   &		& Primární  & ne	& A		&       & Primární klíč \\ \hline
    nazev	        & Varchar	& 20	&	        & ne	&		&       & Název spoje \\ \hline
    cena\_za\_km	& Integer	&	    &	        & ne	&	    & 3     & Cena za 1 ujetý km \\ \hline
    kapacita\_mist	& Integer	&	    &	        & ne	&		&       & Kapacita míst \\ \hline
    pravidelny	    & Boolean	&		&           & ne	&		&       & Pravidelnost \\ \hline
    spolecnost\_id	& Integer	&	    & Cizí	    & ne	&		&       & Společnost \\ \hline
    \end{tabular}
\end{table}

\noindent
Tabulka \textbf{Spolecnost}

\begin{table}[H]
    \begin{tabular}{|l|c|c|c|c|c|c|c|} \hline
                        & Dat. Typ  & Délka & Klíč      & Null  & Index & IO    & Popis \\ \hline
        spolecnost\_id	& Integer	&	    & Primární	& ne	& A		&       & Primární klíč \\ \hline
        nazev	        & Varchar	& 20	&	        & ne	&		&       & Název společnosti \\ \hline
        web             & Varchar   & 30    &           & ne    &       &       & Webová stránka společnosti \\ \hline
        email           & Varchar   & 30    &           & ne    &       &       & Email společnosti \\ \hline
    \end{tabular}
\end{table}

\noindent
Tabulka \textbf{Jizda}

\begin{table}[H]
    \begin{tabular}{|l|c|c|c|c|c|c|c|} \hline
                    & Dat. Typ  & Délka & Klíč      & Null  & Index & IO    & Popis \\ \hline
        jizda\_id	& Integer	&	    & Primární	& ne	& A		&       & Primární klíč \\ \hline
        datum\_start& Date		&	    &           & ne	&		&       & Datum jízdy \\ \hline
        datum\_cil	& Date		&	    &           & ne	&		&       & Datum jízdy \\ \hline
        spoj\_id	& Integer	&	    & Cizí	    & ne	&   	&	    & Spoj \\ \hline
    \end{tabular}
\end{table}

\noindent
Tabulka \textbf{Jizdenka\_Jizda}

\begin{table}[H]
    \begin{tabular}{|l|c|c|c|c|c|c|c|} \hline
                        & Dat. Typ  & Délka & Klíč              & Null  & Index & IO    & Popis \\ \hline
    jizdenka\_id        & Integer   &       & Primární, Cizí    & ne    & A     &       & Jízdenka \\ \hline
    jizda\_id           & Integer   &       & Primární, Cizí	& ne	& A     &       & Jízda \\ \hline
    stanice\_id\_start  & Integer   &       & Cizí              & ne    &       &       & Startovací stanice \\ \hline
    stanice\_id\_cil    & Integer   &       & Cizí              & ne    &       &       & Cílová stanice \\ \hline
    poradi              & Integer   &       &                   & ne    &       &       & Pořadí spoje v jízdence \\ \hline
    \end{tabular}
\end{table}

\noindent
Tabulka \textbf{Jizdenka}

\begin{table}[H]
    \begin{tabular}{|l|c|c|c|c|c|c|c|} \hline
                            & Dat. Typ  & Délka & Klíč      & Null  & Index & IO    & Popis \\ \hline
        jizdenka\_id	    & Integer	&	    & Primární	& ne	& A		&       & Primární klíč \\ \hline
        uzivatel\_id	    & Integer	&	    & Cizí	    & ne	&		&       & Uživatel \\ \hline
        cena	            & Integer	&	    & 	        & ne	&		&       & Cena jízdenky \\ \hline
    \end{tabular}
\end{table}

\newpage

\noindent
Tabulka \textbf{Uzivatel}

\begin{table}[H]
    \begin{tabular}{|l|c|c|c|c|c|c|c|} \hline
                            & Dat. Typ  & Délka & Klíč      & Null  & Index & IO    & Popis \\ \hline
        uzivatel\_id	    & Integer	&	    & Primární	& ne	& A		&       & Primární klíč \\ \hline
        login	            & Varchar	& 20	& 	        & ne	&	    & 	    & Login uživatele použí-\\ &&&&&&& vaný při přihlašování \\ \hline
        jmeno	            & Varchar	& 20	& 	        & ne	&	    & 	    & Jméno uživatele \\ \hline
        prijmeni	        & Varchar	& 20	& 	        & ne	&	    & 	    & Příjmení uživatele \\ \hline
        email	            & Varchar	& 30	& 	        & ne	&	    & 	    & Email uživatele \\ \hline
        typ	                & Varchar	& 20	& 	        & ne	&	    & 1     & Kategorie uživatele \\ \hline
        posledni\_navsteva	& Datetime	& 		&           & ano	&		&       & Datum poslední\\ &&&&&&& návštěvy IS \\ \hline
    \end{tabular}
\end{table}

\noindent
Tabulka \textbf{Historie\_ceny}

\begin{table}[H]
    \begin{tabular}{|l|c|c|c|c|c|c|c|} \hline
                    & Dat. Typ  & Délka & Klíč      & Null  & Index & IO    & Popis \\ \hline
        history\_id	& Integer	&	    & Primární	& ne	& A		&       & Primární klíč \\ \hline
        cena	    & Integer	&	    & 	        & ne	&		&       & Cena spoje \\ \hline
        datum	    & Datetime	&	    & 	        & ne	&		&       & Datum, do kterého byla\\ &&&&&&& hodnota aktuální \\ \hline
        spoj\_id	& Integer	&	    & 	        & ne	&		&       & Spoj \\ \hline
    \end{tabular}
\end{table}

\subsection{Integritní omezení}
\begin{enumerate}
    \item \textit{typ} musí mít hodnotu \textit{spravce drah}, \textit{vlakova spolecnost} nebo \textit{zakaznik}
    \item \textit{vzdalenost} $\geq$ 0
    \item \textit{cena\_za\_km} $>$ 0
\end{enumerate}

\section{Stavová analýza}
Definujeme tyto stavy jízdy:
\begin{itemize}
    \item \textbf{Vytvořená} – vložená jízda
    \item \textbf{Běžící} – vytvořená jízda kde datum konání jízdy \textit{Jizda.datum} je stejné jako aktuální datum a zároveň první příjezd této jízdy má čas \textit{Prijezd.cas} $\geq$ aktuální čas
    \item \textbf{Dokončená} – běžící jízda kde \textit{Jizda.datum} $<$ aktuální datum nebo \textit{Jizda.datum} je stejné jako aktuální datum a poslední příjezd patřící k této jízdě má \textit{Prijezd.cas} $<$ než je aktuální
\end{itemize}

\noindent
Z pohledu vlakové společnosti a uživatele definujeme tyto typy jízd:
\begin{itemize}
    \item \textbf{Vlastní} – jízda patřící ke spoji vytvořená vlakovou společností (\textit{Spoj.spolecnost\_id})
    \item \textbf{Vyprodaná} – jízda které patří počet jízdenek $\geq$ \textit{Spoj.kapacita\_mist}
\end{itemize}

\section{Funkční analýza}

\subsection{Seznam funkcí}

\renewcommand{\labelenumii}{\theenumii}
\renewcommand{\theenumii}{\theenumi.\arabic{enumii}.}

\begin{enumerate}
    % uzivatele
    \item \textbf{Evidence uživatelů}\\
        \textit{Tabulka:} Uzivatel\\
        \textit{Zodpovědnost:} Admin; Zákazník a Vlaková společnost pouze svůj záznam
        \begin{enumerate}
            \item \textbf{Zaregistrování nového uživatele}
            \item \textbf{Aktualizování uživatele}
            \item \textbf{Zrušení uživatele}
            \item \textbf{Seznam uživatelů}
            \item \textbf{Detail uživatele}\\
            \textit{Tabulky:} Uzivatel, Jizdenka
        \end{enumerate}

    % jizdy
    \item \textbf{Evidence jízd}\\
        \textit{Tabulka:} Jizda, Spoj, Prijezd, Stanice\\
        \textit{Zodpovědnost:} Vlaková společnost
        \begin{enumerate}
            \item \textbf{Vytvoření nové jízdy}
            \item \underline{\textbf{Aktualizování jízdy}} – aktualizovat je možné jen jízdu, která ještě nezačala
            \item \textbf{Zrušení jízdy}
            \item \underline{\textbf{Vyhledání jízdy}} – dle startovní/cílové stanice, času odjezdu a s jedním nebo žádným přestupem\\
            \textit{Zodpovědnost:} všichni
            \item \textbf{Detail jízdy}\\
            \textit{Zodpovědnost:} všichni
            \item \underline{\textbf{Vypočítání ceny jízdy}} - podle délky trasy
        \end{enumerate}

    % jizdenky
    \item \textbf{Evidence jízdenek}\\
        \textit{Tabulka:} Jizdenka, Uzivatel, Jizdenka\_Jizda, Jizda, Spoj, Stanice, Mesto\\
        \textit{Zodpovědnost:} všichni
        \begin{enumerate}
            \item \textbf{Vytvoření jízdenky}
            \item \underline{\textbf{Zapsání jízdy do jízdenky}} – uživatel si nemůže objednat jízdu do plného vlaku
            \item \underline{\textbf{Zrušení jízdenky}} – uživatel nemůže zrušit jízdenku, pokud zbývá méně než 15 minut do odjezdu
            \item \textbf{Seznam jízdenek} - zobrazí jízdenky patřící konkrétnímu uživateli
            \item \textbf{Detail jízdenky}\\
            \textit{Tabulky:} všechny\\
            \textit{Zodpovědnost:} všichni
        \end{enumerate}
        
\newpage

    % spoje
    \item \textbf{Evidence spojů}\\
        \textit{Tabulka:} Spoj, Prijezd, Stanice, Mesto, Spolecnost\\
        \textit{Zodpovědnost:} Admin, Vlaková společnost
        \begin{enumerate}
            \item \textbf{Vytvoření nového spoje}
            \item \textbf{Aktualizování spoje} – zapsání původní ceny do tabulky Historie\_ceny
            \item \textbf{Zrušení spoje}
            \item \textbf{Seznam spojů} - filtrace dle stanic kterými spoje projíždí\\
            \textit{Zodpovědnost:} všichni
            \item \textbf{Detail spoje}\\
            \textit{Tabulky:} všechny\\
            \textit{Zodpovědnost:} všichni
        \end{enumerate}

    % prijezdy
    \item \textbf{Evidence příjezdů}\\
        \textit{Tabulka:} Prijezd, Spoj, Stanice, Mesto, Spolecnost\\
        \textit{Zodpovědnost:} Admin, Vlaková společnost
        \begin{enumerate}
            \item \textbf{Vytvoření nového příjezdu}
            \item \textbf{Aktualizování příjezdu}
            \item \textbf{Zrušení příjezdu}
            \item \textbf{Seznam příjezdů} - filtrace podle stanice, spoje nebo data a času příjezdu\\
            \textit{Zodpovědnost:} všichni
            \item \textbf{Detail příjezdu}\\
            \textit{Tabulky:} všechny\\
            \textit{Zodpovědnost:} všichni
        \end{enumerate}

    % stanice
    \item \textbf{Evidence stanic}\\
        \textit{Tabulka:} Stanice\\
        \textit{Zodpovědnost:} všichni
        \begin{enumerate}
            \item \textbf{Seznam stanic}
            \item \textbf{Detail stanice}\\
            \textit{Tabulky:} Stanice, Mesto
        \end{enumerate}

    % mesto
    \item \textbf{Evidence měst}\\
        \textit{Tabulka:} Mesto\\
        \textit{Zodpovědnost:} všichni
        \begin{enumerate}
            \item \textbf{Seznam měst}
            \item \textbf{Detail města}
        \end{enumerate}

    % spolecnosti
    \item \textbf{Evidence společností}\\
        \textit{Tabulka:} Spolecnost\\
        \textit{Zodpovědnost:} všichni
        \begin{enumerate}
            \item \textbf{Seznam společností}
            \item \textbf{Detail společnosti}
        \end{enumerate}
\end{enumerate}

\subsection{Detailní popis funkcí}

% ----------------------------- Aktualizování jízdy -----------------------------

\subsubsection*{2.2 Aktualizování jízdy}
Vstup: \textit{\$p\_jizda\_id}, \textit{\$p\_novy\_datum}, \textit{\$p\_novy\_spoj\_id}\\
Funkce zkontroluje, zda jízda, kterou chceme aktualizovat již neproběhla nebo zrovna neprobíhá. Pokud neproběhla nebo neprobíhá, změní příslušné parametry podle vstupních parametrů \textit{\$p\_novy\_datum} a \textit{\$p\_novy\_spoj\_id}. Jinak vypíše chybové hlášení.

\begin{enumerate}
    \item Do proměnné \textit{\$v\_jizda} uložíme jízdu, kterou chceme aktualizovat:
    \begin{lstlisting}
    SELECT * FROM Jizda WHERE jizda_id = $p_jizda_id;
    \end{lstlisting}

    \item Do proměnné \textit{\$v\_start\_cas} uložíme čas prvního příjezdu spoje:
    \begin{lstlisting}
    SELECT p.cas FROM Prijezd p
        JOIN Spoj s ON p.spoj_id = s.spoj_id
        JOIN Jizda j ON s.spoj_id = j.spoj_id
    WHERE p.spoj_id = $v_jizda.spoj_id AND
        j.datum_start = $v_jizda.datum_start AND
        p.poradi IN (SELECT MIN(poradi) FROM prijezd WHERE spoj_id = $v_jizda.spoj_id);
    \end{lstlisting}
    
    \item Pokud \textit{\$v\_jizda.datum} $>$ aktuální datum nebo \textit{\$v\_jizda.datum} = aktuální datum a zároveň \textit{\$v\_start\_cas} $<$ aktuální čas. Aktualizujeme záznam:
    \begin{lstlisting}
    UPDATE Jizda
    SET Jizda.datum_start = $p_novy_datum,
        Jizda.spoj_id = $p_novy_spoj_id
    WHERE Jizda.jizda_id = $v_jizda.jizda_id;
    \end{lstlisting}

    \item Jinak vypíšeme hlášku „Jízda se již nedá aktualizovat“, vyvoláme výjimku, která zamezí vložení záznamu a proceduru ukončíme.
\end{enumerate}

% ----------------------------- Vyhledání jízdy ----------------------------

\subsubsection*{2.4 Vyhledání jízdy}
Vstup: \textit{\$p\_start\_stanice\_id}, \textit{\$p\_cil\_stanice\_id}, \textit{\$p\_datum}, \textit{\$p\_cas\_od}\\
Funkce najde všechny jízdy s jedním nebo žádným přestupem, které se konají v den zapsaný v proměnné \textit{\$p\_datum} a v čase stejném nebo pozdějším než \textit{\$p\_cas\_od}. Dále jízdy musí projíždět stanicemi \textit{\$p\_start\_stanice\_id} a \textit{\$p\_cil\_stanice\_id}. U cílové stanice ještě zkontrolujeme zda ve spoji následuje po startovní stanici. Funkce vrací řetězec s nalezenými spoji.

\begin{enumerate}
    \item Pomocí následujícího dotazu zjistíme kolik spojů projíždí stanicí \textit{\$p\_start\_stanice\_id} odpovídající časové podmínce. Výsledek zapíšeme do proměnné \textit{\$v\_pocet}:
    \begin{lstlisting}
    SELECT COUNT(DISTINCT p.spoj_id)
    FROM Prijezd p
        JOIN Spoj s ON p.spoj_id = s.spoj_id
        JOIN Jizda j ON s.spoj_id = j.spoj_id
    WHERE p.stanice_id = $p_start_stanice_id AND 
        p.cas >= $p_cas_od AND 
        j.datum = $p_datum;
    \end{lstlisting}
    
    \item V cyklu, který se bude konat \textit{\$v\_pocet}-krát, budeme kontrolovat jestli nám funkce \mbox{\textit{NajdiPrimySpoj()}} najde přímé spoje ze stanice \textit{\$p\_start\_stanice\_id} do stanice \\ \textit{\$p\_cil\_stanice\_id} odpovídající časové podmínce.
    
    \begin{enumerate}
        \item Pokud ano, na výstup pošleme výsledek. Např.:
        \begin{lstlisting}
    Spoj LE 400; Z Ostrava-Svinov; Odjezd 14:12 06.12.19;
                Do Praha hl.n.; Prijezd 17:23
    Spoj RJ 106; Z Ostrava-Svinov; Odjezd 15:12 06.12.19;
                Do Praha hl.n.; Prijezd 18:23
        \end{lstlisting}

        \item Pokud ne, pokusíme se najít spoj s přestupem. Do proměnné \textit{\$v\_prvni\_spoj} budeme v každé iteraci cyklu ukládat \textit{spoj\_id} \textit{i}-tého řádku z následujícího dotazu. Spoj musí projíždět startovní stanici a splňovat podmínky data a času:
        \begin{lstlisting}
    SELECT DISTINCT p.spoj_id, p.cas
    FROM Prijezd p
        JOIN Spoj s ON p.spoj_id = s.spoj_id
        JOIN Jizda j ON s.spoj_id = j.spoj_id
    WHERE p.stanice_id = $p_start_stanice_id AND
        p.cas >= $p_cas_od AND 
        j.datum = $p_datum
    ORDER BY p.cas
        \end{lstlisting}

        \item Následně do proměnné \textit{\$v\_pocet\_stanic} uložíme počet stanic spoje \textit{\$v\_prvni\_spoj}:
        \begin{lstlisting}
    SELECT COUNT(stanice_id) FROM Prijezd
    WHERE spoj_id = $v_prvni_spoj;
        \end{lstlisting}

        \item Abychom mohli zjistit jestli se můžeme z nějaké stanice na spoji \textit{\$v\_prvni\_spoj} dostat do cílové stanice přestupem na jiný spoj. Budeme ve vnořeném cyklu, který se bude provádět \textit{\$v\_pocet\_stanic}-krát v každé iteraci ukládat hodnotu \textit{stanice\_id} \textit{j}-tého řádku do proměnné \textit{\$v\_prestupni\_stanice} pomocí dotazu:
        \begin{lstlisting}
        SELECT stanice_id, cas FROM Prijezd
        WHERE spoj_id = v_prvni_spoj
        ORDER BY cas
        \end{lstlisting}

        \begin{enumerate}
            \item Pokud funkce \textit{NajdiPrimySpoj()} najde přímý spoj z \textit{\$v\_prestupni\_stanice} do \textit{\$p\_cil\_stanice\_id}, pošleme na výstup oba navazující spoje. Např.:
            \begin{lstlisting}
    Spoj LE 900; Bohumin - Odjezd 06.12.20 14:12; Hranice na Morave - Prijezd 14:30; Prestup na spoj RJ 800; Odjezd - 14:45; Zabreh na Morave Prijezd 15:05
    Spoj LE 901; Bohumin - Odjezd 06.12.20 15:12; Hranice na Morave - Prijezd 15:30; Prestup na spoj RJ 802; Odjezd - 15:45; Zabreh na Morave Prijezd 16:05
            \end{lstlisting}
        \end{enumerate}
    \end{enumerate}
\end{enumerate}

\newpage
% ----------------------------- Vypočítání ceny jízdy ----------------------------

\subsubsection*{2.6 Vypočítání ceny jízdy}
Vstup: \textit{\$p\_stanice\_id\_start}, \textit{\$p\_stanice\_id\_cil}, \textit{\$p\_jizda\_id}\\
Funkce vypočítá cenu jízdenky podle vstupních parametrů a vrátí výsledek.

\begin{enumerate}
    \item Do proměnné \textit{\$v\_start\_vzdalenost} uložíme vzdálenost dle \textit{\$p\_stanice\_id\_start} a \textit{\$p\_jizda\_id}:
    \begin{lstlisting}
    SELECT DISTINCT vzdalenost
    FROM prijezd
        JOIN spoj ON prijezd.spoj_id = spoj.spoj_id
        JOIN jizda ON spoj.spoj_id = jizda.spoj_id
    WHERE prijezd.stanice_id = $p_stanice_id_start AND jizda.jizda_id = $p_jizda_id;
    \end{lstlisting}
    
    \item Do proměnné \textit{\$v\_cil\_vzdalenost} uložíme vzdálenost dle \textit{\$p\_stanice\_id\_cil} a \textit{\$p\_jizda\_id}:
    \begin{lstlisting}
    SELECT DISTINCT vzdalenost
    FROM prijezd
        JOIN spoj ON prijezd.spoj_id = spoj.spoj_id
        JOIN jizda ON spoj.spoj_id = jizda.spoj_id
    WHERE prijezd.stanice_id = $p_stanice_id_cil AND jizda.jizda_id = $p_jizda_id;
    \end{lstlisting}

    \item Do proměnné \textit{\$v\_cena\_za\_km} uložíme cenu za kilometr dle \textit{\$p\_jizda\_id}:
    \begin{lstlisting}
    SELECT DISTINCT cena_za_km
    FROM spoj
        JOIN jizda ON spoj.spoj_id = jizda.spoj_id
    WHERE jizda.jizda_id = $p_jizda_id;
    \end{lstlisting}

    \item Vrátíme výsledek výrazu:
    \begin{lstlisting}
    ($v_cil_vzdalenost - $v_start_vzdalenost) * $v_cena_za_km;
    \end{lstlisting}
\end{enumerate}

% ----------------------------- Zapsání jízdy do jízdenky ----------------------------

\subsubsection*{3.2	Zapsání jízdy do jízdenky}
Vstup: \textit{\$p\_jizdenka\_id}, \textit{\$p\_jizda\_id}, \textit{\$p\_stanice\_id\_start}, \textit{\$p\_stanice\_id\_cil}\\
Funkce zkontroluje, zda zbývají volná místa ve spoji na který je vázaná jízda \textit{\$p\_jizda\_id}. Pokud ano, jizda se zapíše do jízdenky. Jinak vypíše chybové hlášení.

\begin{enumerate}
    \item Do proměnné \textit{\$v\_pocet\_jizdenek} uložíme vypočtenou hodnotu celkového počtu objednaných jízdenek pro \textit{\$p\_jizda\_id}:
    \begin{lstlisting}
    SELECT COUNT(jizdenka_id) FROM jizdenka_jizda
    WHERE jizda_id = $p_jizda_id;
    \end{lstlisting}

    \item Do proměnné \textit{\$v\_pocet\_mist} uložíme počet celkových míst v našem spoji:
    \begin{lstlisting}
    SELECT kapacita_mist FROM Spoj s
        JOIN Jizda j ON s.spoj_id = j.spoj_id
    WHERE jizda_id = $p_jizda_id;
    \end{lstlisting}

    \item Pokud \textit{\$v\_pocet\_jizdenek} $<$ \textit{\$v\_pocet\_mist}, přidáme záznam do tabulky \textit{Jizdenka\_Jizda}:
    \begin{lstlisting}
    INSERT INTO jizdenka_jizda(jizdenka_id, jizda_id, stanice_id_start, stanice_id_cil, poradi)
    VALUES ($p_jizdenka_id, $p_jizda_id, $p_stanice_id_start, $p_stanice_id_cil, 
        (SELECT COALESCE(MAX(poradi), 0) FROM jizdenka_jizda WHERE jizdenka_id = p_jizdenka_id) + 1);
    \end{lstlisting}

    \item Pokud nebude volný dostatečný počet míst procedura vypíše: „Spoj je vyprodaný“ a bude ukončena.
\end{enumerate}

% ----------------------------- Zrušení jízdenky ----------------------------

\subsubsection*{3.3 Zrušení jízdenky}
Vstup: \textit{\$p\_jizdenka\_id}\\
Funkce zkontroluje, zda nezbývá 15 minut do odjezdu vlaku. Pokud zbývá více než 15 minut, funkce smaže jízdenku. Jinak vypíše chybové hlášení.

\begin{enumerate}
    \item Zjistíme startovní stanici a datum a čas kdy se koná první jízda v jízdence. Proto do proměnné \textit{\$v\_jizda\_datum} a \textit{\$v\_prvni\_stanice} uložíme první výsledek následujícího dotazu:
    \begin{lstlisting}
    SELECT datum_start, stanice_id_start
    FROM Jizda j
        JOIN jizdenka_jizda jj ON j.jizda_id = jj.jizda_id
    WHERE jj.jizdenka_id = $p_jizdenka_id
    ORDER BY datum_start
    \end{lstlisting}
    
    \item Dále do proměnné \textit{\$v\_cas\_odjezdu} uložíme první výsledek následujícího dotazu:
    \begin{lstlisting}
    SELECT cas
    FROM Prijezd p
        JOIN Spoj s ON p.spoj_id = s.spoj_id
        JOIN Jizda j ON s.spoj_id = j.spoj_id
        JOIN jizdenka_jizda jj ON j.jizda_id = jj.jizda_id
    WHERE jj.jizdenka_id = $p_jizdenka_id AND
        p.stanice_id = $v_prvni_stanice AND
        j.datum_start = $v_jizda_datum
    ORDER BY cas
    \end{lstlisting}

    \item Pokud (\textit{\$v\_jizda\_datum} = aktualni datum a zároveň (\textit{\$v\_cas\_odjezdu} – aktuální čas) $>$ 15 minut) nebo \textit{\$v\_jizda\_datum} $>$ aktuální datum. Smažeme jízdenku:
    \begin{lstlisting}
    DELETE FROM jizdenka_jizda
        WHERE jizdenka_id = $p_jizdenka_id;
    DELETE FROM Jizdenka
        WHERE jizdenka_id = $p_jizdenka_id;
    \end{lstlisting}

    \item Jinak vypíšeme chybové hlášení: „Jízdenka již nelze zrušit“, vyvoláme výjimku a proceduru ukončíme.
\end{enumerate}

% ----------------------------- Uzivatelske rozhrani ----------------------------

\newpage

\section{Návrh uživatelského rozhraní}

\subsection{Menu}

\begin{enumerate}
    \item \textbf{Můj profil} – akce: 1.5 Detail uživatele\\
        \textit{Zodpovědnost}: všichni
        \begin{enumerate}
            \item \textbf{Upravit profil} - akce: 1.2 Aktualizování uživatele
            \item \textbf{Smazat profil} - akce: 1.3 Zrušení uživatele
        \end{enumerate}

    \item \textbf{Najít spojení} - akce: 2.4 Vyhledání jízdy\\
        \textit{Zodpovědnost}: Zákazník

    \item \textbf{Moje jízdenky} - akce: 3.4 Seznam jízdenek\\
        \textit{Zodpovědnost}: Zákazník
        \begin{enumerate}
            \item \textbf{Storno jízdenky} - akce: 3.3 Zrušení jízdenky
        \end{enumerate}
    
    \item \textbf{Najít vlak} - akce: 4.4 Seznam spojů\\
        \textit{Zodpovědnost}: Zákazník

    \item \textbf{Přehled příjezdů} - akce: 5.4 Seznam příjezdů\\
        \textit{Zodpovědnost}: všichni

    \item \textbf{Seznam stanic} - akce: 6.1 Seznam stanic\\
        \textit{Zodpovědnost}: všichni
        \begin{enumerate}
            \item \textbf{Detail stanice} - akce: 6.2 Detail stanice
        \end{enumerate}

    \item \textbf{Vlaková společnost}
        \textit{Zodpovědnost}: Admin, Vlaková společnost
        \begin{enumerate}
            \item \textbf{Správa spojů}
            \item \textbf{Správa jízd}
            \item \textbf{Správa příjezdů}
        \end{enumerate}
    
    \item \textbf{Administrace}
        \textit{Zodpovědnost}: Admin
        \begin{enumerate}
            \item \textbf{Správa uživatelů}
            \item \textbf{Správa společností}
            \item \textbf{Správa stanic}
            \item \textbf{Správa měst}
        \end{enumerate}
\end{enumerate}

% ----------------------------- Návrh formulářů ----------------------------

\subsection{Návrh formulářů}

\subsubsection*{2.4 Vyhledání jízdy}

\begin{figure}[H]
    \includegraphics[width=.80\textwidth]{formular3.png}
    \centering
    \caption{Vyhledání jízdy}
    \label{vystup}
\end{figure}

\subsubsection*{2.5 Detail jízdy}

\begin{figure}[H]
    \includegraphics[width=\textwidth]{formular1.png}
    \centering
    \caption{Detail jízdy}
    \label{vystup}
\end{figure}

\subsubsection*{3.3 Seznam jízdenek}

\begin{figure}[H]
    \includegraphics[width=\textwidth]{formular2.png}
    \centering
    \caption{Seznam jízdenek}
    \label{vystup}
\end{figure}

%}
\end{document}
